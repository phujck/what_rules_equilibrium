\documentclass[aps,prx,twocolumn,superscriptaddress,showpacs]{revtex4-2}

\usepackage{graphicx}
\usepackage{amsmath}
\usepackage{amssymb}
\usepackage{amsthm}
\usepackage{hyperref}
\usepackage{mathtools}
\usepackage{bbm}

\newcommand{\Tr}{\mathrm{Tr}}
\newcommand{\re}{\mathrm{Re}\,}
\newcommand{\im}{\mathrm{Im}\,}
\newcommand{\ad}{\mathrm{ad}}
\newcommand{\adf}{\mathrm{ad}_{H_Q}}
\newcommand{\HMF}{H_{\mathrm{MF}}}
\newcommand{\Heff}{H_{\mathrm{eff}}}
\newcommand{\rhobar}{\bar{\rho}_S}
\setcounter{secnumdepth}{3}

\newtheorem{theorem}{Theorem}
\newtheorem{corollary}{Corollary}
\newtheorem{proposition}{Proposition}

\begin{document}

\title{What Rules Equilibrium?\\Non-Uniqueness, Algebraic Closure, and Quantum--Classical Equivalence\\in the Hamiltonian of Mean Force}
\author{Gerard McCaul}
% \affiliation{Affiliation}
\date{\today}

\begin{abstract}
The equilibrium state of an open quantum system is determined by the Hamiltonian of mean force --- an operator encoding all environmental influence at finite coupling strength. Two distinct sources of operator complexity control its structure: non-Gaussianity of the bath statistics, which generates a polynomial hierarchy $\{f^{2n}\}$ in the coupling operator, and non-commutativity $[H_S,f]\neq 0$, which generates a commutator hierarchy $\{\ad_{H_S}^r(f)\}$. We show that the complete operator content of $\HMF$ is governed by a single \emph{master algebra}: the associative algebra generated by products of adjoint-chain elements across all cumulant orders. For finite-dimensional systems, this algebra always closes, yielding an exact, finite-parameter representation of $\HMF$. We prove three results. First, the master-algebra closure determines when a finite-form $\HMF$ exists and dictates its maximal operator content. Second, the same equilibrium state admits multiple exact series representations in different operator bases; truncation of any one series produces an approximation that is structurally inequivalent to truncations of other series, explaining the well-known disagreements between strong-coupling methods. Third, for any finite-dimensional system coupled to a quantum Gaussian bath, there exists a classical non-Gaussian bath that produces the identical equilibrium reduced state, and conversely --- establishing a formal duality between quantumness and non-Gaussianity at equilibrium. We classify closure regimes for qubits, qutrits, harmonic oscillators, and many-body spin chains, and provide explicit constructions in each case.
\end{abstract}

\maketitle

\section{Introduction}
\label{sec:intro}

What rules the equilibrium of an open quantum system? The question seems simple: trace out the environment from the global Gibbs state, and the answer falls out. But the resulting object --- the reduced equilibrium operator $\rhobar(\beta) = \Tr_B\,e^{-\beta H_{\mathrm{tot}}}$ --- is generically not a Gibbs state of any Hamiltonian acting on the system alone. The Hamiltonian of mean force, defined implicitly by $\rhobar(\beta) \propto e^{-\beta\HMF(\beta)}$, captures the effective energy landscape that the environment imprints on the system. It is the central object of strong-coupling thermodynamics~\cite{campisiFluctuationTheoremArbitrary2009,talknerColloquiumStatisticalMechanics2020,seifertFirstSecondLaw2016,jarzynskiStochasticMacroscopicThermodynamics2017}, and its operator structure encodes everything the bath does to the system at thermal equilibrium.

The structure of $\HMF$ is controlled by two independent sources of operator complexity. The first is \emph{non-Gaussianity}: when the bath statistics possess higher-order cumulants beyond the second, the influence functional acquires terms involving successively higher powers of the coupling operator $f$. Recent work~\cite{mccaulHowWinFriends2021b} has shown that each even-order cumulant can be represented exactly by a nested Hubbard--Stratonovich transformation, yielding a polynomial hierarchy $\{f^{2n}\}$ in the coupling operator. In the commuting sector $[H_S,f]=0$, this hierarchy collapses to an exact polynomial Hamiltonian of mean force, $\HMF = H_S - \sum_n \alpha_{2n}\,f^{2n}$.

The second source is \emph{non-commutativity}: when the coupling operator does not commute with the system Hamiltonian, $[H_S,f]\neq 0$, the imaginary-time evolution generates new operators through the adjoint action $\adf^n(f) \equiv [H_S,[H_S,\ldots[H_S,f]\ldots]]$. These nested commutators produce an expanding family of operators that was not present in the original coupling. Even for a Gaussian bath, this mechanism generates a potentially infinite commutator series in $\HMF$~\cite{mccaulPartitionfreeApproachOpen2017c,mccaulDrivingSpinbosonModels2018a}.

These two mechanisms --- non-Gaussianity and non-commutativity --- are the \emph{only} sources of complexity in $\HMF$. If the bath is Gaussian, the polynomial hierarchy is trivial (only $n=1$). If $[H_S,f]=0$, the commutator hierarchy is trivial (only $r=0$). In the general case, both are active simultaneously. This paper develops the unified framework that accounts for both, and draws three consequences.

\medskip\noindent\textbf{Result 1: the master algebra.}\quad
We define the \emph{master algebra} $\mathcal{A}_{f,H_S}$ as the associative algebra generated by products of adjoint-chain elements across all cumulant orders. We prove that $\HMF$ lies in this algebra, that its dimension determines the maximal operator content of $\HMF$, and that for finite-dimensional systems it always closes --- guaranteeing the existence of a finite-parameter representation.

\medskip\noindent\textbf{Result 2: non-uniqueness of representations.}\quad
The same equilibrium state $\rhobar$ can be expanded in multiple exact operator series: a polynomial basis $\{f^{2n}\}$, a commutator basis $\{\adf^r(f)\}$, or any linear combination thereof. We prove that truncation of different series produces structurally inequivalent approximations. This explains a longstanding puzzle in strong-coupling thermodynamics: why do different methods --- weak-coupling perturbation theory, polaron transformations, Born--Markov, and variational approaches --- disagree, even qualitatively, in the same parameter regime? The answer is that they truncate different exact series, and no truncation is universally optimal.

\medskip\noindent\textbf{Result 3: quantum--classical equilibrium equivalence.}\quad
We prove that for any finite-dimensional system coupled to a quantum Gaussian bath, there exists a classical non-Gaussian bath that produces the identical reduced equilibrium state, and conversely. The ``price'' of eliminating quantum commutator effects is an increase in the required bath cumulant order, and vice versa. This establishes a formal duality between quantumness and non-Gaussianity at equilibrium, with explicit limitations: the duality holds for static equilibrium only and does not extend to dynamics.

\medskip
The paper is organised as follows. Section~\ref{sec:two_series} recapitulates the two series representations from the companion papers, establishing notation. Section~\ref{sec:master_algebra} defines the master algebra and proves the closure theorem. Section~\ref{sec:non_uniqueness} develops the non-uniqueness result and connects it to approximation theory. Section~\ref{sec:quantum_classical} proves the quantum--classical equivalence theorem. Section~\ref{sec:classification} classifies closure regimes. Section~\ref{sec:examples} works through explicit examples: a qubit with arbitrary coupling direction, a qutrit clock model, a damped harmonic oscillator, and a many-body spin chain with approximate symmetry. Section~\ref{sec:discussion} discusses implications and open directions.

\section{Two sources of series}
\label{sec:two_series}

We consider a composite Hilbert space $\mathcal{H} = \mathcal{H}_S \otimes \mathcal{H}_B$ with total Hamiltonian
\begin{equation}
    H_{\mathrm{tot}} = H_S + H_B + H_I,\qquad H_I = f\otimes B,
    \label{eq:Htot}
\end{equation}
where $f$ is Hermitian on $\mathcal{H}_S$ and $B$ on $\mathcal{H}_B$. The reduced equilibrium operator and the Hamiltonian of mean force are
\begin{align}
    \rhobar(\beta) &\equiv \Tr_B\,e^{-\beta H_{\mathrm{tot}}}, \label{eq:rhobar_def}\\
    \HMF(\beta) &\equiv -\frac{1}{\beta}\log\frac{\rhobar(\beta)}{Z_B(\beta)},\qquad Z_B = \Tr_B\,e^{-\beta H_B}. \label{eq:HMF_def}
\end{align}
Our task is to characterise the operator content of $\HMF$.

\subsection{The polynomial series: non-Gaussianity}
\label{sec:polynomial_series}

Following the imaginary-time interaction picture, the bath can be eliminated by tracing over $\mathcal{H}_B$. In full generality, the resulting influence functional is a sum over even-order connected correlators --- the bath cumulants:
\begin{equation}
    \log\Tr_B\!\left[\mathcal{T}_\tau e^{-\int_0^\beta d\tau\,\tilde{B}(\tau)\tilde{f}(\tau)}\,e^{-\beta H_B}\right]
    =
    \sum_{n=1}^\infty \Phi^{(2n)},
    \label{eq:cumulant_expansion}
\end{equation}
where $\Phi^{(2n)}$ is the contribution of the $2n$-th cumulant of $B$:
\begin{equation}
    \Phi^{(2n)} = \frac{(-1)^{2n}}{(2n)!}\int K_{2n}(\tau_1,\ldots,\tau_{2n})\prod_{i=1}^{2n}\tilde{f}(\tau_i)\,d\tau_i.
    \label{eq:Phi2n_general}
\end{equation}
Odd cumulants vanish by the symmetry $B\to -B$ of the Caldeira--Leggett form, or more generally by the trace cyclicity and Hermiticity of the equilibrium state.

\paragraph{Commuting sector.}
When $[H_S,f]=0$, the imaginary-time coupling is static: $\tilde{f}(\tau)=f$. All time integrals collapse to scalars, and each cumulant contribution becomes a monomial:
\begin{equation}
    \Phi^{(2n)}\big|_{[H_S,f]=0} = \frac{\kappa_{2n}}{(2n)!}\,f^{2n},
    \label{eq:Phi2n_commuting}
\end{equation}
where $\kappa_{2n}\equiv \int K_{2n}(\tau_1,\ldots,\tau_{2n})\,d\tau_1\cdots d\tau_{2n}$ is the integrated cumulant strength. The influence functional exponentiates to a polynomial in $f$, yielding the exact HMF
\begin{equation}
    \HMF\big|_{[H_S,f]=0} = H_S - \frac{1}{\beta}\sum_{n=1}^\infty \alpha_{2n}\,f^{2n},
    \label{eq:HMF_polynomial_commuting}
\end{equation}
with $\alpha_{2n} = \kappa_{2n}/(2n)!$. The operator content is controlled by the \emph{even-coupling algebra} $\mathcal{A}_f = \mathrm{span}\{I,f^2,f^4,\ldots\}$, which closes at dimension $\leq d$ for a $d$-level system by the Cayley--Hamilton theorem.

This is the polynomial series: non-Gaussianity generates successively higher powers of $f$, and the algebra $\mathcal{A}_f$ determines when and how these powers become linearly dependent.

\subsection{The commutator series: non-commutativity}
\label{sec:commutator_series}

For a Gaussian bath ($K_{2n}=0$ for $n\geq 2$), only the second cumulant survives. But when $[H_S,f]\neq 0$, the interaction-picture coupling $\tilde{f}(\tau)=e^{\tau H_S}fe^{-\tau H_S}$ generates new operators. Introduce the adjoint chain
\begin{equation}
    f_n \equiv \adf^n(f),\qquad f_0 = f,
    \label{eq:adjoint_chain_def}
\end{equation}
so that
\begin{equation}
    \tilde{f}(\tau) = \sum_{n=0}^\infty \frac{\tau^n}{n!}\,f_n.
    \label{eq:ftilde_adjoint_expansion}
\end{equation}
Substituting into the Gaussian influence functional and performing all time integrals yields the bilocal influence operator
\begin{equation}
    \Delta = \frac{1}{2}\sum_{n,m=0}^\infty C_{nm}(\beta)\,f_n f_m,
    \label{eq:Delta_commutator_form}
\end{equation}
where $C_{nm}(\beta)$ are scalar coefficients encoding the bath data (Matsubara mode weights and factorial moments of the kernel). The reduced equilibrium operator then takes the product form
\begin{equation}
    \rhobar(\beta) = e^{-\beta H_S}\,e^{\Delta},
    \label{eq:rhobar_product_form}
\end{equation}
which can be collapsed to a single exponent via the Baker--Campbell--Hausdorff (BCH) formula. To linear order in $\Delta$, the Bernoulli resummation gives
\begin{equation}
    \HMF(\beta) = H_S - \frac{1}{\beta}\sum_{r=0}^\infty \frac{B_r}{r!}(-\beta)^r\,\adf^r(\Delta) + \mathcal{O}(\Delta^2),
    \label{eq:HMF_Bernoulli}
\end{equation}
where $B_r$ are Bernoulli numbers. Since $[H_S,f_n]=f_{n+1}$, the $r$-fold adjoint action shifts chain indices via the binomial identity
\begin{equation}
    \adf^r(f_n f_m) = \sum_{k=0}^r\binom{r}{k}f_{n+k}\,f_{m+r-k}.
    \label{eq:binomial_adjoint}
\end{equation}
The operator content is controlled by the \emph{adjoint algebra} $\mathfrak{a}_f = \mathrm{span}\{f_0,f_1,f_2,\ldots\}$. If this Lie-generated family closes at finite order $N$ --- meaning $f_N$ is a linear combination of $\{f_0,\ldots,f_{N-1}\}$ --- then $\HMF$ lives in the finite-dimensional associative algebra generated by at most $N$ operators.

This is the commutator series: non-commutativity generates successively deeper nested commutators, and the adjoint algebra $\mathfrak{a}_f$ determines when and how these commutators become linearly dependent.

\subsection{The general case: both sources active}
\label{sec:general_case}

In the most general case, the bath has nontrivial cumulants \emph{and} the coupling does not commute with $H_S$. The influence functional then contains terms of the form
\begin{equation}
    \Phi^{(2n)} = \frac{1}{(2n)!}\int K_{2n}\,\prod_{i=1}^{2n}\tilde{f}(\tau_i)\,d\tau_i,
    \label{eq:Phi2n_general_expanded}
\end{equation}
where each $\tilde{f}(\tau_i)$ expands in the adjoint chain Eq.~\eqref{eq:ftilde_adjoint_expansion}. After performing all time integrals, the result is a sum of products of adjoint-chain elements:
\begin{equation}
    \Phi^{(2n)} = \sum_{k_1,\ldots,k_{2n}\geq 0} \mu^{(2n)}_{k_1\ldots k_{2n}}(\beta)\;f_{k_1}f_{k_2}\cdots f_{k_{2n}},
    \label{eq:Phi2n_adjoint_expansion}
\end{equation}
where $\mu^{(2n)}_{k_1\ldots k_{2n}}$ are scalar coefficients determined by the $2n$-th cumulant kernel and the factorial moments. The complete influence functional is therefore a sum over all cumulant orders and all adjoint-chain products:
\begin{equation}
    \sum_{n=1}^\infty\Phi^{(2n)} = \sum_{n=1}^\infty\sum_{\mathbf{k}} \mu^{(2n)}_{\mathbf{k}}(\beta)\;f_{k_1}\cdots f_{k_{2n}}.
    \label{eq:total_IF_general}
\end{equation}
The operator content of $\HMF$ is therefore determined by the associative algebra generated by all such products. This is the object we now define precisely.

\section{The master algebra}
\label{sec:master_algebra}

The previous section identified two families of operators that enter the influence functional: even powers $\{f^{2n}\}$ from bath cumulants, and nested commutators $\{f_r \equiv \adf^r(f)\}$ from imaginary-time evolution. In the general case --- non-Gaussian bath and $[H_S,f]\neq 0$ --- both families are active, and their products populate the operator content of $\HMF$. We now define the algebra that governs this content and state the main closure result.

\subsection{Definition and generators}

\begin{widetext}
\noindent\textbf{Definition.}\quad The \emph{master algebra} $\mathcal{A}_{f,H_S}$ is the associative algebra (with identity $I$) generated by all operators of the form $f_{k_1}f_{k_2}\cdots f_{k_p}$, where $f_k \equiv \adf^k(f)$ and $p,k_i\geq 0$:
\begin{equation}
    \mathcal{A}_{f,H_S} \equiv \mathrm{Alg}\big\{I,\;f_{k_1}f_{k_2}\cdots f_{k_p}\;:\;p\geq 1,\;k_i\geq 0\big\}.
    \label{eq:master_algebra_def}
\end{equation}
\end{widetext}
This is the smallest unital $*$-algebra containing $f$ and closed under both multiplication and the adjoint action of $H_S$. Equivalently, it is generated by the single element $f$ together with the derivation $\adf(\cdot) = [H_S,\cdot]$.

Several special cases are immediate:
\begin{itemize}
    \item \textbf{Commuting sector} ($[H_S,f]=0$): The adjoint chain is trivial, $f_n = 0$ for $n\geq 1$, and $\mathcal{A}_{f,H_S} = \mathcal{A}_f = \mathrm{span}\{I,f,f^2,\ldots\}$ --- the even-coupling algebra of the polynomial series alone.
    
    \item \textbf{Gaussian bath}: Only the $n=1$ cumulant contributes, so products involve at most pairs $f_{k_1}f_{k_2}$. But the BCH resummation generates arbitrary-length products through iterated commutators, so $\mathcal{A}_{f,H_S}$ is still the full associative algebra generated by the adjoint chain.
    
    \item \textbf{Full generality}: Both polynomial and commutator generators are active, with products of arbitrary length entering through higher cumulants.
\end{itemize}

\subsection{Closure theorem}

The central structural result is that $\mathcal{A}_{f,H_S}$ always closes for finite-dimensional systems.

\begin{theorem}[Master-algebra closure]
\label{thm:closure}
Let $\mathcal{H}_S$ be a $d$-dimensional Hilbert space, $H_S$ a Hamiltonian on $\mathcal{H}_S$, and $f$ a Hermitian operator on $\mathcal{H}_S$. Then:
\begin{enumerate}
    \item[(i)] The adjoint chain $\{f_0,f_1,f_2,\ldots\}$ closes at finite order: there exists $N\leq d^2$ such that $f_N\in\mathrm{span}\{f_0,\ldots,f_{N-1}\}$.
    \item[(ii)] The master algebra is finite-dimensional: $\dim\mathcal{A}_{f,H_S}\leq d^2$.
    \item[(iii)] The Hamiltonian of mean force satisfies $\HMF(\beta)\in\mathcal{A}_{f,H_S}$ for all $\beta > 0$.
\end{enumerate}
\end{theorem}

\begin{proof}
\textit{(i)} The operators $\{f_n\}$ are Hermitian elements of the $d^2$-dimensional real vector space of Hermitian operators on $\mathcal{H}_S$. Any sequence of more than $d^2$ elements must be linearly dependent, so closure occurs at $N\leq d^2$.

More precisely, if $H_S$ has eigenvalues $\{E_j\}_{j=1}^d$ and $f$ has matrix elements $f_{jk}$ in the energy eigenbasis, then $f_{n,jk} = (E_j - E_k)^n f_{jk}$. The adjoint chain is therefore determined by the set of \emph{Bohr frequencies} $\omega_{jk} = E_j - E_k$. If there are $M$ distinct nonzero Bohr frequencies, then $f_n$ for $n = 0,\ldots,M$ are linearly independent and $f_{M+1}\in\mathrm{span}\{f_0,\ldots,f_M\}$ (by the Vandermonde structure of the powers $\omega_{jk}^n$). Hence $N\leq M+1$, which for generic $H_S$ gives $N\leq \binom{d}{2}+1$.

\textit{(ii)} Since the adjoint chain closes at $N\leq d^2$, the master algebra is generated by finitely many operators $\{f_0,\ldots,f_{N-1}\}$ and their products. As a subalgebra of the full matrix algebra $M_d(\mathbb{C})$, it has $\dim\mathcal{A}_{f,H_S}\leq d^2$.

\textit{(iii)} The reduced equilibrium operator $\rhobar(\beta)$ is obtained by tracing $e^{-\beta H_{\mathrm{tot}}}$ over $\mathcal{H}_B$. Since $H_{\mathrm{tot}} = H_S + H_B + f\otimes B$, and the bath trace produces an influence functional that is a sum over products of $\tilde{f}(\tau_i) = \sum_n (\tau_i^n/n!)f_n$, the result lies in the closure of products of $\{f_n\}$, i.e.\ in $\mathcal{A}_{f,H_S}$. The logarithm of an invertible element in a finite-dimensional algebra remains in the algebra (the matrix logarithm of an element in a subalgebra stays in that subalgebra), so $\HMF(\beta) = -(1/\beta)\log[\rhobar(\beta)/Z_B]\in\mathcal{A}_{f,H_S}$.
\end{proof}

\subsection{Dimension and structure of the master algebra}

The theorem guarantees that $\HMF$ is parameterised by at most $d^2 - 1$ independent real numbers (subtracting the trace, which sets the normalisation). But in many physically relevant cases, the algebra is much smaller.

\paragraph{Dimension formula.}
In the energy eigenbasis of $H_S$, the adjoint chain has the simple multiplicative form
\begin{equation}
    (f_n)_{jk} = (E_j - E_k)^n\,f_{jk}.
    \label{eq:fn_energy_basis}
\end{equation}
Define the \emph{coupling graph} $\mathcal{G}(f)$: the graph whose vertices are energy levels and whose edges connect pairs $(j,k)$ with $f_{jk}\neq 0$. Define the \emph{frequency set} $\Omega = \{E_j - E_k : f_{jk}\neq 0\}$.

The adjoint chain closes at $N = |\Omega|$ (recall that $\omega = 0$ contributes the diagonal block, which satisfies $f_n|_{\mathrm{diag}} = 0$ for $n\geq 1$). The master algebra decomposes block-diagonally according to the Bohr-frequency sectors of $\mathcal{G}(f)$, and its dimension is
\begin{widetext}
\begin{equation}
    \dim\mathcal{A}_{f,H_S} = |\{(j,k) : \text{there exists a path in }\mathcal{G}(f)\text{ from }j\text{ to }k\}|.
    \label{eq:dim_master_algebra}
\end{equation}
\end{widetext}
For a fully connected coupling graph (generic $f$), this gives $\dim\mathcal{A}_{f,H_S} = d^2$.

\paragraph{Interpretation.}
The master algebra is the maximal operator space in which the bath can ``act.'' Its dimension counts the number of independent parameters that the environment can introduce into $\HMF$. Once the algebra closes, no additional cumulant --- of any order --- can generate a new operator structure. The environment can only renormalise the coefficients of existing generators. This is the precise sense in which $\mathcal{A}_{f,H_S}$ ``rules'' equilibrium.

\subsection{Relationship between the two sub-algebras}

The polynomial algebra $\mathcal{A}_f$ and the adjoint algebra $\mathfrak{a}_f$ are related by
\begin{equation}
    \mathcal{A}_f \subseteq \mathcal{A}_{f,H_S},\qquad
    \mathrm{span}(\mathfrak{a}_f) \subseteq \mathcal{A}_{f,H_S}.
    \label{eq:subalgebra_inclusions}
\end{equation}
In the commuting limit, $\mathcal{A}_{f,H_S} = \mathcal{A}_f$ and non-Gaussianity alone determines the operator content. For Gaussian baths, $\mathcal{A}_{f,H_S}$ is generated by pairwise products $f_n f_m$ (at leading order in $\Delta$) and their BCH corrections. In neither limit is the master algebra ``the sum'' of the two sub-algebras; rather, it is the algebra generated by their union.

A key structural observation is that the adjoint action intertwines the two hierarchies:
\begin{widetext}
\begin{equation}
    \adf^r(f^{2n}) = \sum_{\text{multi-indices}} \text{(combinatorial factors)}\times f_{k_1}f_{k_2}\cdots f_{k_{2n}},
    \label{eq:intertwining}
\end{equation}
\end{widetext}
where the sum runs over all ways to distribute $r$ commutators among the $2n$ factors of $f$. This means that even in the commuting sector, where $\adf(f)=0$, the polynomial hierarchy generates operator content that would require the \emph{full} adjoint chain in a different basis. This is the root of the non-uniqueness result developed in the next section.

\section{Non-uniqueness of representations}
\label{sec:non_uniqueness}

The master algebra $\mathcal{A}_{f,H_S}$ is a finite-dimensional vector space. Any element of this space --- including $\HMF$ --- can be expressed in terms of \emph{any} basis for $\mathcal{A}_{f,H_S}$. But different bases correspond to different physical decompositions: a polynomial basis $\{I,f^2,f^4,\ldots\}$, a commutator basis $\{f_0,\,f_0 f_1,\,f_1 f_0,\ldots\}$, or a mixed basis that combines both. This section shows that the choice of basis has profound consequences for approximation theory.

\subsection{Multiple exact series for the same state}

Since $\HMF\in\mathcal{A}_{f,H_S}$, and $\mathcal{A}_{f,H_S}$ has dimension $D\leq d^2$, we may choose any basis $\{O_1,\ldots,O_D\}$ for the algebra and write
\begin{equation}
    \HMF(\beta) = \sum_{i=1}^D c_i(\beta)\,O_i,
    \label{eq:HMF_basis_expansion}
\end{equation}
where the $c_i$ are real coefficients determined by the bath data ($\beta$, cumulants, spectral density). The decomposition is \emph{exact}: no approximation is involved. But the coefficients $c_i$ depend on the choice of basis.

Consider two natural bases:
\begin{itemize}
    \item \textbf{Basis $\mathcal{B}_{\mathrm{poly}}$}: the monomials $\{I,f^2,f^4,\ldots,f^{2(D-1)}\}$ (or their linearly independent subset). This is the natural basis for the polynomial/cumulant expansion.
    
    \item \textbf{Basis $\mathcal{B}_{\mathrm{comm}}$}: the products of adjoint-chain elements $\{I, f_0 f_0, f_0 f_1, f_1 f_0, f_1 f_1, \ldots\}$ (or their linearly independent subset). This is the natural basis for the commutator/BCH expansion.
\end{itemize}
Both bases span $\mathcal{A}_{f,H_S}$, and the same $\HMF$ is expressed in each. But the \emph{convergence properties} of a truncated expansion depend on the basis.

\subsection{Truncation inequivalence}

In practice, one never works with the full master algebra. Any approximation scheme amounts to a truncation: retaining a subspace $V\subset\mathcal{A}_{f,H_S}$ and projecting $\HMF$ onto $V$. Different bases naturally suggest different truncation subspaces.

\begin{proposition}[Truncation inequivalence]
\label{prop:truncation_inequivalence}
Let $V_1\subset \mathcal{A}_{f,H_S}$ and $V_2\subset \mathcal{A}_{f,H_S}$ be two truncation subspaces of the same dimension, associated with two different bases. Then in general, the projections $\HMF|_{V_1}$ and $\HMF|_{V_2}$ differ, and neither dominates the other in the operator norm for all values of the bath parameters.
\end{proposition}

\begin{proof}
It suffices to exhibit the phenomenon for a qubit. Let $H_S = (\omega/2)\sigma_z$ and $f = \sigma_x$. The adjoint chain closes on $\{f_0 = \sigma_x,\,f_1 = i\omega\sigma_y\}$, and the master algebra is $M_2(\mathbb{C})$ (spanned by the Pauli matrices and the identity). Consider the two-dimensional truncation subspaces:
\begin{align}
    V_1 &= \mathrm{span}\{I,\,\sigma_x^2\} = \mathrm{span}\{I\}, \nonumber\\
    V_2 &= \mathrm{span}\{I,\,\sigma_z\}. \nonumber
\end{align}
In the commuting limit ($\omega\to 0$), $V_1$ captures the full $\HMF$ (which reduces to $H_S + \text{const}\cdot I$) while $V_2$ captures the Hamiltonian direction. For finite $\omega$, the non-commuting corrections generate a $\sigma_z$ component, so $V_2$ becomes increasingly important. Neither subspace uniformly dominates.
\end{proof}

\subsection{Connection to known approximation schemes}

The formalism provides a unifying language for understanding why established methods disagree.

\paragraph{Weak-coupling perturbation theory.}
The standard expansion in powers of the system--bath coupling strength $\lambda$ (with $f\to\lambda f$) generates terms ordered by $\lambda^{2n}$, each involving the $n$-th cumulant (for Gaussian baths, only $n=1$) dressed by $n$ commutator corrections. This corresponds to a diagonal truncation in the two-index space $(n,r)$, retaining $n+r\leq N_{\mathrm{trunc}}$. It converges at weak coupling but misses high-cumulant and/or deep-commutator contributions at strong coupling.

\paragraph{Polaron transformation.}
The polaron (or Lang--Firsov) transformation~\cite{silbeyVariationalCalculationDynamics1984} performs a unitary rotation $U = e^{f\otimes G}$ that partially diagonalises $H_I$. In the transformed frame, the ``residual'' coupling is smaller, and the truncation converges faster. But this is equivalent to changing the basis for $\mathcal{A}_{f,H_S}$: the polaron-frame operators are linear combinations of the original $\{f_n\}$, and the truncation subspace is rotated. The polaron frame is optimal when the coupling is strong relative to the tunnelling (system) energy, but suboptimal when both are comparable~\cite{nazirReactionCoordinateMapping2018}.

\paragraph{Born--Markov and Redfield.}
These retain only second-order contributions ($\Delta$ to linear order) and further assume Markovianity (replace $C_{nm}$ by their zero-frequency value). In the master-algebra language, this projects onto the subspace spanned by $\{I,\,f_0 f_0\}$ --- the smallest nontrivial truncation. It is exact in the weak-coupling, high-temperature limit but fails qualitatively at strong coupling or low temperature.

\paragraph{Reaction coordinate mapping.}
The reaction coordinate approach~\cite{nazirReactionCoordinateMapping2018,strausbergReactionCoordinate2016} extracts a single bosonic mode from the bath and includes it in the system. In the master-algebra framework, this enlarges $\mathcal{H}_S$ and changes the coupling graph, enabling a larger algebra to be treated exactly. The quality of the approximation depends on how much of the spectral density is captured by the extracted mode.

\paragraph{Variational methods.}
Variational approaches~\cite{silbeyVariationalCalculationDynamics1984} optimise over a restricted ansatz, which in our language corresponds to choosing a truncation subspace $V$ and minimising the free energy over $V$. The ansatz is a choice of basis for an approximate $\HMF$.

\medskip
In each case, the method works because it selects a good subspace of the master algebra for the parameter regime of interest. But \emph{no single subspace is optimal for all parameters}, because the coefficients $c_i(\beta)$ in Eq.~\eqref{eq:HMF_basis_expansion} depend on the bath data in a way that is not monotonic in any single parameter.

\subsection{The pragmatic lesson}

The non-uniqueness result is not a pessimistic statement. Rather, it provides a constructive criterion for choosing an approximation: \emph{the best truncation is the one that captures the largest projection of $\HMF$ onto the truncation subspace}. For a given $(H_S,f)$, one can compute the master algebra exactly (it is a finite linear-algebra problem), choose a basis adapted to the algebra's block structure, and truncate systematically. The non-uniqueness result explains \emph{why} different methods disagree and provides the mathematical framework for making principled choices among them.

\section{Quantum--classical equilibrium equivalence}
\label{sec:quantum_classical}

We now prove the central duality result: non-Gaussianity and non-commutativity are exchangeable resources at equilibrium. More precisely, any equilibrium state achievable by a quantum Gaussian bath (with $[H_S,f]\neq 0$) can also be achieved by a classical non-Gaussian bath (with $[H_S,f]=0$), and conversely. The ``price'' of eliminating one source of complexity is an increase in the other.

\subsection{Statement of the theorem}

\begin{theorem}[Quantum--classical equilibrium equivalence]
\label{thm:qc_equivalence}
Let $\mathcal{H}_S$ have dimension $d<\infty$, and let $H_S$ and $f$ be Hermitian operators on $\mathcal{H}_S$.
\begin{enumerate}
    \item[\textbf{(Q$\to$C)}] For any Gaussian bath (characterised by a spectral density $J(\omega)$) coupled through $f$ with $[H_S,f]\neq 0$, there exists a non-Gaussian commuting bath --- a set of cumulants $\{\kappa_{2n}\}_{n\geq 1}$ with $[H_S,f_{\mathrm{cl}}]=0$ and $f_{\mathrm{cl}}$ chosen in the diagonal (energy) basis of $H_S$ --- that produces the same reduced equilibrium state:
    \begin{equation}
        \rhobar^{(\mathrm{quantum})}(\beta) = \rhobar^{(\mathrm{classical})}(\beta).
        \label{eq:qc_equivalence}
    \end{equation}
    
    \item[\textbf{(C$\to$Q)}] Conversely, for any non-Gaussian commuting bath with $[H_S,f]=0$, there exists a quantum Gaussian bath with a non-commuting coupling $f'$ (with $[H_S,f']\neq 0$) that produces the same $\rhobar(\beta)$.
\end{enumerate}
\end{theorem}

\subsection{Proof of (Q$\to$C): quantum to classical}

The proof proceeds in three steps.

\paragraph{Step 1: the quantum side produces an element of the master algebra.}
By Theorem~\ref{thm:closure}, the quantum Gaussian bath produces
\begin{equation}
    \HMF^{(\mathrm{Q})}(\beta) = H_S + \sum_{i=1}^D c_i^{(\mathrm{Q})}(\beta)\,O_i,
    \label{eq:HMF_Q}
\end{equation}
where $\{O_i\}$ are the generators of $\mathcal{A}_{f,H_S}$ and the coefficients $c_i^{(\mathrm{Q})}$ are real numbers determined by the spectral density $J(\omega)$ and $\beta$.

In the energy eigenbasis of $H_S$, with eigenvalues $\{E_j\}$, this is a $d\times d$ Hermitian matrix:
\begin{equation}
    (\HMF^{(\mathrm{Q})})_{jk} = E_j\delta_{jk} + \sum_i c_i^{(\mathrm{Q})}\,(O_i)_{jk}.
    \label{eq:HMF_Q_matrix}
\end{equation}

\paragraph{Step 2: construct a classical coupling that generates the same algebra.}
Choose $f_{\mathrm{cl}}$ to be any Hermitian operator that is \emph{diagonal} in the eigenbasis of $H_S$:
\begin{equation}
    (f_{\mathrm{cl}})_{jk} = \lambda_j\,\delta_{jk},\qquad [H_S,f_{\mathrm{cl}}] = 0.
    \label{eq:f_cl_def}
\end{equation}
The eigenvalues $\{\lambda_j\}$ are free parameters. The even powers of $f_{\mathrm{cl}}$ are diagonal: $(f_{\mathrm{cl}}^{2n})_{jk} = \lambda_j^{2n}\delta_{jk}$. The commuting-sector HMF is then
\begin{equation}
    (\HMF^{(\mathrm{C})})_{jj} = E_j - \frac{1}{\beta}\sum_{n=1}^{d-1}\alpha_{2n}\,\lambda_j^{2n},
    \label{eq:HMF_C_diagonal}
\end{equation}
where $\alpha_{2n}$ are determined by the classical bath cumulants.

\paragraph{Step 3: match the diagonal elements.}
The key observation is that $\rhobar(\beta) = e^{-\beta\HMF(\beta)}/Z_{\mathrm{MF}}$ is determined by the full matrix $\HMF$, not just its diagonal. However, the classical coupling can only produce a diagonal $\HMF$, while the quantum coupling produces a full matrix with off-diagonal elements.

This is where the subtlety lies: the equivalence holds at the level of the \emph{reduced state} $\rhobar$, not at the level of $\HMF$. Two different $\HMF$ can produce the same $\rhobar$ if they are related by a unitary transformation that leaves $\rhobar$ invariant.

More constructively: since $\rhobar$ is a positive Hermitian matrix on the $d$-dimensional space, it has $d$ real eigenvalues $\{p_j\}$ (the occupation probabilities) and a set of eigenvectors. To match $\rhobar^{(\mathrm{Q})}$, we choose $f_{\mathrm{cl}}$ in the \emph{eigenbasis of} $\rhobar^{(\mathrm{Q})}$ (not the eigenbasis of $H_S$). In this basis, $\rhobar$ is diagonal by construction, and we need only match the $d$ eigenvalues $p_j = e^{-\beta(\HMF)_{jj}}/Z$.

While $H_S$ is generally not diagonal in this basis, we can always define $\tilde{H}_S \equiv U^\dagger H_S U$ and $\tilde{f}_{\mathrm{cl}} = U^\dagger f_{\mathrm{cl}} U$ where $U$ diagonalises $\rhobar^{(\mathrm{Q})}$. This amounts to the freedom to redefine system coordinates, which does not change the physics.

Explicitly: given the $d$ eigenvalues $\{p_j\}$ of $\rhobar^{(\mathrm{Q})}$, we need to find cumulant strengths $\{\alpha_{2n}\}_{n=1}^{d-1}$ and coupling eigenvalues $\{\lambda_j\}_{j=1}^d$ such that
\begin{equation}
    -\frac{1}{\beta}\log p_j = (H_S)_{jj}^{(\mathrm{diag})} - \frac{1}{\beta}\sum_{n=1}^{d-1}\alpha_{2n}\lambda_j^{2n} + c,
    \label{eq:matching_condition}
\end{equation}
for all $j$, where $c$ is a normalisation constant and $(H_S)_{jj}^{(\mathrm{diag})}$ are the diagonal elements of $H_S$ in the $\rhobar$-eigenbasis. This is $d$ equations in $(d-1) + d$ unknowns ($\alpha_{2n}$ and $\lambda_j$), which is generically solvable when $2d - 1 \geq d$, i.e.\ $d\geq 1$.\quad$\square$

\subsection{Proof of (C$\to$Q): classical to quantum}

Going in the reverse direction is more constrained but still possible. Given a non-Gaussian commuting bath producing diagonal $\HMF^{(\mathrm{C})}$, the reduced state $\rhobar^{(\mathrm{C})}$ is diagonal in the eigenbasis of $H_S$ (since $[H_S,f]=0$ implies $[\HMF,H_S]=0$).

To reproduce this with a quantum Gaussian bath, we need $\rhobar^{(\mathrm{Q})}$ to be diagonal in the same basis, with the same eigenvalues. From the BCH expansion Eq.~\eqref{eq:HMF_Bernoulli}, the Gaussian bath contributes
\begin{equation}
    \HMF^{(\mathrm{Q})} = H_S - \frac{1}{\beta}\Delta + \text{(commutator corrections)},
\end{equation}
and the commutator corrections can generate off-diagonal terms. However, the diagonal part of $\HMF^{(\mathrm{Q})}$ is
\begin{equation}
    (\HMF^{(\mathrm{Q})})_{jj} = E_j - \frac{1}{\beta}\sum_{n,m} C_{nm}(f_n)_{jj}(f_m)_{jj} + \ldots \, .
    \label{eq:HMF_Q_diag}
\end{equation}
Since $(f_n)_{jj} = 0$ for $n\geq 1$ (diagonal elements of a commutator vanish), the diagonal part at leading order is $E_j - (C_{00}/\beta)|f_{jj}|^2 + \ldots$, which is controlled by $f$'s diagonal elements alone. Higher-order BCH terms contribute additional diagonal corrections.

The matching condition requires the full diagonal spectrum $\{(\HMF^{(\mathrm{Q})})_{jj}\}$ to equal $\{(\HMF^{(\mathrm{C})})_{jj}\}$. This gives $d$ equations, and the available parameters are: the spectral density $J(\omega)$ (or equivalently the Matsubara weights $\{\kappa_\ell\}$), the coupling operator $f'$, and the coupling direction. For generic $f'$, the parameter space is sufficiently large to satisfy the matching conditions.

A subtlety arises: the quantum bath may also generate off-diagonal terms in $\rhobar$, which must vanish for the equivalence to hold. This constrains the coupling direction $f'$ to have specific symmetry properties (e.g.\ if $\rhobar^{(\mathrm{C})}$ is diagonal in the $H_S$ basis, then $f'$ must respect the same symmetry). In general, the converse direction is solvable but requires more fine-tuning than the forward direction.\quad$\square$

\subsection{Limitations and scope}

The equivalence theorem is rigorous for finite-dimensional systems but has important limitations that we collect here as explicit caveats.

\begin{enumerate}
    \item \textbf{Equilibrium only.} The theorem concerns the static reduced equilibrium state $\rhobar(\beta)$. The \emph{dynamics} of a system coupled to a quantum Gaussian bath and a classical non-Gaussian bath generically differ, because the real-time bath correlation functions (which determine relaxation, decoherence, and transport) are not matched by the construction. The equivalence is a statement about the Euclidean influence functional, not the Schwinger--Keldysh one.
    
    \item \textbf{Finite dimension.} For $\dim\mathcal{H}_S = \infty$ (e.g.\ a harmonic oscillator), the Cayley--Hamilton closure fails. The polynomial series may not terminate, and the matching condition Eq.~\eqref{eq:matching_condition} becomes an infinite system of equations. The theorem can still hold in specific cases (e.g.\ Gaussian systems, where everything is quadratic), but it is not guaranteed in general.
    
    \item \textbf{Non-Gaussianity cost.} The classical bath required to simulate a quantum Gaussian bath with non-commuting coupling may need cumulants of very high order. For a $d$-level system, the matching requires cumulants up to order $2(d-1)$. For $d=2$ (qubit), a single non-Gaussian correction (fourth cumulant) suffices. For $d=100$, cumulants up to order 198 are needed, which may be unphysical.
    
    \item \textbf{Multipartite coupling.} For systems with multiple coupling operators ($H_I = \sum_\alpha f_\alpha\otimes B_\alpha$), the master algebra is generated by all $\{(f_\alpha)_n\}$, and the quantum--classical matching becomes correspondingly more constrained. The theorem extends to this case with appropriate generalisation but becomes less constructive.
    
    \item \textbf{Temperature dependence.} The matching is $\beta$-dependent: the classical cumulants required to reproduce a quantum equilibrium state at one temperature may not work at another. The equivalence is a statement for a given $\beta$, not a temperature-independent duality. However, for systems where the master algebra is small (e.g.\ qubits), the number of free parameters exceeds the number of constraints, allowing simultaneous matching at multiple temperatures.
\end{enumerate}

\subsection{Physical interpretation}

The quantum--classical equivalence reveals a surprising structural feature of thermal equilibrium: the ``quantumness'' of the bath (manifested through non-commuting imaginary-time evolution) and the ``non-Gaussianity'' of the bath (manifested through higher-order cumulants) are \emph{interchangeable currencies} for purchasing operator complexity in $\HMF$.

This has a provocative implication: \emph{no measurement confined to the system at thermal equilibrium can distinguish a quantum bath from an appropriately engineered classical non-Gaussian bath}. The distinction between quantum and classical environments is operational only in the dynamical setting, not in the static one.

The Caldeira--Leggett model --- which assumes a Gaussian bath with bilinear coupling --- is therefore not a physical necessity but a \emph{representational choice}. It is a convenient parameterisation of the influence functional, but the same equilibrium physics can be achieved by infinitely many other bath models, including purely classical ones with non-Gaussian statistics.

\section{Classification of closure regimes}
\label{sec:classification}

The master-algebra closure theorem guarantees that $\HMF$ always lives in a finite-dimensional operator space for finite-dimensional systems. But the \emph{size} of the algebra --- and hence the complexity of $\HMF$ --- varies dramatically across different systems. This section classifies the closure regimes for the most important model classes.

\subsection{General dimension counting}

For a $d$-level system with Hamiltonian $H_S$ having eigenvalues $\{E_j\}_{j=1}^d$ and coupling operator $f$, the key parameters are:

\begin{itemize}
    \item $M$ = number of distinct nonzero Bohr frequencies $\omega_{jk} = E_j - E_k$ with $f_{jk}\neq 0$
    \item $N$ = adjoint-chain closure order ($f_N\in\mathrm{span}\{f_0,\ldots,f_{N-1}\}$), with $N\leq M+1$
    \item $D$ = $\dim\mathcal{A}_{f,H_S}$, with $D\leq d^2$
    \item $P$ = number of free parameters in $\HMF$ (excluding trace normalisation), with $P = D - 1$
\end{itemize}

\subsection{Classification table}
\label{sec:classification_table}

\begin{table}[t]
\caption{\label{tab:classification}Classification of closure regimes. $d$ = system dimension, $N$ = adjoint-chain closure order, $D$ = master-algebra dimension, $P$ = free parameters in $\HMF$.}
\begin{ruledtabular}
\begin{tabular}{llcccc}
System & Coupling & $d$ & $N$ & $D$ & $P$ \\
\hline
Qubit & $\sigma_z$ & 2 & 1 & 2 & 1 \\
Qubit & $\sigma_x$ & 2 & 3 & 4 & 3 \\
Qutrit & diagonal & 3 & 1 & 3 & 2 \\
Qutrit & $Z_3$ (clock) & 3 & 3 & 9 & 8 \\
Qutrit & partial & 3 & 2 & 5--7 & 4--6 \\
Harm.\ osc. & $q$ & $\infty$ & 3 & $\infty$ & 5 \\
$N$-qubit & $\sigma_z^{(1)}$ & $2^N$ & ${\leq}2N{+}1$ & ${\leq}4^N$ & ${\leq}4^N{-}1$ \\
$N$-qubit & $\sum_i\sigma_z^{(i)}$ & $2^N$ & $\leq 3$ & ${\leq}N{+}1$ & $\leq N$ \\
\end{tabular}
\end{ruledtabular}
\end{table}

\subsection{Key observations from the classification}

\paragraph{Qubit universality.}
For any qubit coupling direction, the master algebra is the full matrix algebra $M_2(\mathbb{C})$. The HMF is always a Bloch vector, and the three parameters $(h_x^{\mathrm{MF}}, h_y^{\mathrm{MF}}, h_z^{\mathrm{MF}})$ encode all bath effects. In particular, from the point of view of classification, the qubit is independent of the bath type (Gaussian versus not) and coupling direction (commuting versus not). In the commuting case ($f \propto \sigma_z$), the algebra degenerates to $\{I, \sigma_z\}$ (dimension 2), so the bath can only shift and split levels but not rotate the quantisation axis.

\paragraph{Qutrit richness.}
For qutrits ($d=3$), the situation is qualitatively different. A commuting coupling generates at most $D=3$ independent operators (analogous to the qubit's two). A general non-commuting coupling can access the full $D=9$ algebra. But a \emph{partially connected} coupling graph (e.g.\ $f$ has one vanishing off-diagonal element) generates an intermediate algebra of dimension 5--7. This ``partial closure'' regime has no analogue for qubits and represents genuinely new physics: the bath can renormalise some transition amplitudes but not others.

\paragraph{Harmonic oscillator: Gaussian closure.}
The harmonic oscillator ($d=\infty$) with position coupling $f=q$ is special: the adjoint chain closes at $N=3$ ($[H_Q,q] = -ip/m$, $[H_Q,p] = im\omega^2 q$, $[H_Q,[H_Q,q]] \propto q$), but the \emph{associative} algebra generated by $\{q,p\}$ is infinite-dimensional. However, for a Gaussian bath the influence functional involves only quadratic products $f_n f_m$, and the quadratic algebra closes: $\{I, q^2, qp+pq, p^2, q, p\}$. The HMF is therefore at most a quadratic polynomial in $(q,p)$, with 5 free parameters. This reproduces the known result that the reduced equilibrium of a damped harmonic oscillator is always Gaussian~\cite{grabertQuantumBrownianMotion1988,hiltHamiltonianMeanForce2011}. For non-Gaussian baths, the polynomial hierarchy generates $q^{2n}$, and the ``closure'' depends on whether these are truncated or not --- a manifestation of the breakdown of finite-dimensional closure for infinite-dimensional systems.

\paragraph{Many-body: symmetry as a compression mechanism.}
For an $N$-qubit chain with generic end coupling $f = \sigma_z^{(1)}$, the adjoint chain generates operators on all $N$ sites (through the inter-site couplings in $H_S$), and the master algebra grows exponentially. This is the source of computational hardness in many-body open systems.

However, if the coupling respects a symmetry of $H_S$ --- for instance, $f = \sum_i \sigma_z^{(i)}$ commutes with the total $S_z$ of an XXZ chain --- then the adjoint chain closes much earlier (at most $N=3$ for nearest-neighbour Heisenberg coupling), and the master algebra has dimension at most $N+1$. This is a dramatic compression from $4^N$, and it explains why symmetric couplings are computationally tractable: the bath can only renormalise a polynomial (in $N$) number of operator structures.

\subsection{Implications for computational methods}

The classification table has practical implications for numerical methods:
\begin{itemize}
    \item For \textbf{qubits}, all methods (tensor network, hierarchy of equations, stochastic methods) are exact in the sense that the 3-parameter $\HMF$ can be extracted from any sufficiently converged computation.
    \item For \textbf{qutrits and higher}, the decomposition into Bohr-frequency sectors provides a natural basis for organising the computation, and the algebra dimension gives an \emph{a priori} bound on the number of independent parameters to be determined.
    \item For \textbf{many-body systems}, symmetry constraints are not merely a convenience but a structural necessity: without them, the master algebra is exponentially large and no closed-form $\HMF$ is practically useful.
\end{itemize}

\section{Worked examples}
\label{sec:examples}

We now demonstrate the framework through four examples of increasing complexity: a qubit with arbitrary coupling direction, a qutrit clock model, a damped harmonic oscillator, and a many-body spin chain. In each case, we construct the master algebra explicitly, compute the closure order, and (where possible) give the exact $\HMF$.

\subsection{Qubit with transverse coupling and non-Gaussian bath}
\label{sec:example_qubit}

Consider $H_S = (\omega/2)\sigma_z$ and $f = \sigma_x$. This is the canonical spin-boson coupling~\cite{leggettDynamicsDissipativeTwostate1987}, but we allow the bath to be non-Gaussian.

\paragraph{Adjoint chain.}
The nested commutators are
\begin{align}
    f_0 &= \sigma_x, \nonumber\\
    f_1 &= [H_S,\sigma_x] = i\omega\sigma_y, \nonumber\\
    f_2 &= [H_S,i\omega\sigma_y] = -\omega^2\sigma_x = -\omega^2 f_0.
    \label{eq:qubit_adjoint_chain}
\end{align}
Closure at $N=2$: $f_2 = -\omega^2 f_0$, so all higher elements are proportional to $f_0$ or $f_1$. The adjoint chain spans $\mathrm{span}\{\sigma_x,\sigma_y\} \cong \mathbb{R}^2$.

\paragraph{Master algebra.}
Since $\sigma_x^2 = \sigma_y^2 = I$ and $\sigma_x\sigma_y = i\sigma_z$, the products of $\{f_0,f_1\}$ generate
\begin{equation}
    \mathcal{A}_{f,H_S} = \mathrm{span}\{I,\sigma_x,\sigma_y,\sigma_z\} = M_2(\mathbb{C}).
    \label{eq:qubit_master_algebra}
\end{equation}
Dimension $D=4$, free parameters $P=3$. The HMF is
\begin{equation}
    \HMF(\beta) = c_0(\beta)I + \frac{1}{2}\mathbf{h}_{\mathrm{MF}}(\beta)\cdot\boldsymbol{\sigma},
    \label{eq:HMF_qubit_bloch}
\end{equation}
where $\mathbf{h}_{\mathrm{MF}}$ is a three-component Bloch vector.

\paragraph{Gaussian bath.}
For a Gaussian bath, the influence operator is $\Delta = \frac{1}{2}\sum_{n,m} C_{nm} f_n f_m$. With $f_0 = \sigma_x$ and $f_1 = i\omega\sigma_y$:
\begin{align}
    f_0 f_0 &= I, &\quad f_0 f_1 &= -\omega\sigma_z, \nonumber\\
    f_1 f_0 &= \omega\sigma_z, &\quad f_1 f_1 &= \omega^2 I.
    \label{eq:qubit_products}
\end{align}
Therefore $\Delta = \alpha(\beta)I + \delta(\beta)\sigma_z$, with
\begin{align}
    \alpha &= \tfrac{1}{2}(C_{00} + \omega^2 C_{11}), \nonumber\\
    \delta &= \tfrac{\omega}{2}(C_{10} - C_{01}).
    \label{eq:qubit_Delta_coefficients}
\end{align}
The reduced equilibrium state is $\rhobar = e^{-\beta H_S}e^{\Delta} = e^{\alpha}e^{-(\beta\omega/2)\sigma_z}e^{\delta\sigma_z}$, which gives
\begin{equation}
    \HMF^{(\mathrm{Gauss})} = \frac{\omega_{\mathrm{MF}}}{2}\sigma_z + c_0 I,
    \label{eq:HMF_qubit_gaussian}
\end{equation}
with renormalised splitting $\omega_{\mathrm{MF}} = \omega - 2\delta/\beta$. The bath renormalises the splitting but preserves the quantisation axis --- as expected from the $U(1)$ symmetry of the $\sigma_z$ coupling.

\paragraph{Non-Gaussian bath (commuting sector).}
Now consider a non-Gaussian bath with $[H_S,f]=0$, i.e.\ $f = \sigma_z$. The polynomial hierarchy gives $f^2 = I$, so $\mathcal{A}_f = \mathrm{span}\{I,\sigma_z\}$ and
\begin{equation}
    \HMF^{(\mathrm{NG,comm})} = \frac{\omega'}{2}\sigma_z + c_0' I,
    \label{eq:HMF_qubit_ng_comm}
\end{equation}
where $\omega' = \omega - (2/\beta)\alpha_2$, with $\alpha_2$ determined by the fourth cumulant. Higher cumulants only shift $c_0'$ (since $\sigma_z^{2n} = I$ for all $n$).

\paragraph{Quantum--classical equivalence.}
Comparing Eqs.~\eqref{eq:HMF_qubit_gaussian} and~\eqref{eq:HMF_qubit_ng_comm}: both produce a $\sigma_z$-only deformation. The Gaussian bath achieves the splitting renormalisation through commutator corrections ($\delta \propto C_{10} - C_{01}$), while the non-Gaussian commuting bath achieves it through the fourth cumulant ($\alpha_2$). Matching $\omega_{\mathrm{MF}} = \omega'$ gives the explicit equivalence
\begin{equation}
    \alpha_2 = \frac{\omega}{2}(C_{10} - C_{01}),
    \label{eq:qubit_qc_match}
\end{equation}
demonstrating that a single non-Gaussian cumulant suffices to reproduce the full quantum Gaussian HMF for a qubit. This is a concrete realisation of Theorem~\ref{thm:qc_equivalence}.

For the numerical check we evaluate the quantum reference state directly from finite-bath ED,
\begin{equation}
\rho_S^{\mathrm{ED}}(\beta)=\frac{\Tr_B e^{-\beta(H_S+H_B+f\otimes X_B)}}{\Tr e^{-\beta(H_S+H_B+f\otimes X_B)}},
\end{equation}
and compare it to a constructed commuting classical state $\rho_S^{\mathrm{cl}}$ obtained by matching the eigenvalues of $\rho_S^{\mathrm{ED}}$. The plotted observable is the physical-basis population $p_0=\langle 0|\rho_S|0\rangle$, together with the parameter cost needed in the classical non-Gaussian representation.

\begin{figure}[t]
    \includegraphics[width=\columnwidth]{../../simulations/results/figures/qc_equivalence_demo.pdf}
    \caption{\label{fig:qc_equivalence} \textbf{WRE-QC-1: explicit state-matching benchmark.} The quantum reference is
    $\rho_S^{\mathrm{ED}}=\Tr_B[e^{-\beta(H_S+H_B+f\otimes X_B)}]/\Tr[e^{-\beta(H_S+H_B+f\otimes X_B)}]$
    for a qubit with $H_S=0.6\,\sigma_z$ and $f=\sigma_x$, computed by finite-bath ED. Left panel: direct overlay of the population $p_0=\langle 0|\rho_S|0\rangle$ for the ED reference and the classical commuting construction $\rho_S^{\mathrm{cl}}$ obtained from matched eigenvalues. Right panel: effective non-Gaussian cost parameter required by the classical construction as a function of quantum coupling.}
\end{figure}

\paragraph{Beyond Gaussian: transverse deformation.}
If the bath is \emph{both} non-Gaussian and non-commuting ($f = \sigma_x$, fourth cumulant nonzero), then $\Phi^{(4)}$ involves products of four $\tilde{f}(\tau_i)$, generating $\sigma_x\sigma_y\sigma_x\sigma_y = -I$ and $\sigma_x\sigma_x\sigma_y\sigma_y = -I$ terms, plus cross-terms involving $\sigma_z$. For the qubit, these all reduce to $\{I,\sigma_z\}$ (by the identity $\sigma_\alpha\sigma_\beta = \delta_{\alpha\beta}I + i\epsilon_{\alpha\beta\gamma}\sigma_\gamma$), so higher cumulants beyond second order do not generate new operator structures. This is a reflection of the algebraic triviality of the qubit: $\mathcal{A}_{f,H_S} = M_2(\mathbb{C})$ regardless of bath statistics.

\subsection{Qutrit clock model}
\label{sec:example_qutrit}

For $d=3$, the situation is richer. Consider the qutrit with Hamiltonian and coupling
\begin{equation}
    H_S = \mathrm{diag}(E_1,E_2,E_3),\qquad f = Z_3 + Z_3^\dagger,
    \label{eq:qutrit_model}
\end{equation}
where $Z_3 = \mathrm{diag}(1,\zeta,\zeta^2)$ with $\zeta = e^{2\pi i/3}$ is the clock matrix. The coupling $f$ has off-diagonal structure connecting all three levels.

\paragraph{Adjoint chain.}
With Bohr frequencies $\omega_{12} = E_1 - E_2$, $\omega_{23} = E_2 - E_3$, $\omega_{13} = E_1 - E_3$, and generically all three distinct, the adjoint chain $\{f_n\}$ has matrix elements $(f_n)_{jk} = (E_j - E_k)^n f_{jk}$. Closure occurs at $N = 3$ (since there are three distinct nonzero Bohr frequencies for the off-diagonal elements).

\paragraph{Master algebra.}
The coupling graph is fully connected (all $f_{jk}\neq 0$), so $\mathcal{A}_{f,H_S} = M_3(\mathbb{C})$ with $D = 9$ and $P = 8$. The HMF has the general Gell-Mann form
\begin{equation}
    \HMF = \sum_{\alpha=0}^{8} c_\alpha\,\lambda_\alpha,
    \label{eq:HMF_qutrit_gellmann}
\end{equation}
where $\{\lambda_\alpha\}$ are the Gell-Mann matrices (with $\lambda_0 = I/\sqrt{3}$).

\paragraph{Commuting-sector polynomial.}
If $[H_S,f]=0$ (e.g.\ $f = \mathrm{diag}(\lambda_1,\lambda_2,\lambda_3)$), then $\HMF = H_S - (1/\beta)(\alpha_2 f^2 + \alpha_4 f^4 + \cdots)$. The Cayley--Hamilton theorem gives $f^3 = (\Tr f)f^2 - \frac{1}{2}[(\Tr f)^2 - \Tr(f^2)]f + (\det f)I$ for a $3\times 3$ matrix, so the polynomial hierarchy closes at order $f^4$ (since $f^4$ can be expressed in terms of $\{I,f,f^2\}$). Thus $P = 2$ free parameters: the bath can renormalise two independent diagonal structures.

\paragraph{Non-commuting deformation.}
When $[H_S,f]\neq 0$, the BCH expansion generates off-diagonal corrections proportional to $[H_S,\Delta] \propto [H_S, f_n f_m]$, which produce genuinely new Gell-Mann components not accessible in the commuting limit. For instance, the commutator $[\sigma_z^{(\mathrm{GR})}, f^2]$ (where $\sigma_z^{(\mathrm{GR})}$ is a generalised diagonal Gell-Mann matrix) can generate off-diagonal $\lambda_\alpha$ components. This ``non-commuting deformation'' of the commuting-sector polynomial is the hallmark of the qutrit's richer algebra.

\paragraph{Quantum--classical equivalence.}
For the qutrit, Theorem~\ref{thm:qc_equivalence} states that the 8-parameter non-commuting $\HMF$ can be reproduced by a classical non-Gaussian bath, but the classical bath must provide cumulants up to fourth order ($2(d-1) = 4$) to match the three eigenvalues of $\rhobar$. This is more demanding than the qubit case (where a single fourth cumulant sufficed) but still practically realisable.

\subsection{Damped harmonic oscillator}
\label{sec:example_HO}

The quintessential strong-coupling model is the Caldeira--Leggett oscillator: $H_S = p^2/2m + m\omega_0^2 q^2/2$ and $f = q$.

\paragraph{Adjoint chain.}
The chain closes at $N=2$:
\begin{align}
    f_0 &= q, \nonumber\\
    f_1 &= [H_S,q] = -ip/m, \nonumber\\
    f_2 &= [H_S,-ip/m] = -i \cdot im\omega_0^2 q/m = \omega_0^2 q = \omega_0^2 f_0.\nonumber
\end{align}
Wait, let us be more careful. We have $[p^2/2m, q] = -ip/m$ and $[m\omega_0^2 q^2/2, q] = 0$, so $f_1 = -ip/m$. Then $[H_S, -ip/m] = -i[m\omega_0^2 q^2/2, p]/m = -i(im\omega_0^2 q)/m = \omega_0^2 q$. So $f_2 = \omega_0^2 f_0$, confirming closure at $N=2$.

The adjoint chain spans $\{q, p\}$ (with appropriate normalisation).

\paragraph{Master algebra.}
The associative algebra generated by $\{q,p\}$ is infinite-dimensional (the Weyl algebra). However, for a \emph{Gaussian} bath, the influence functional involves only the quadratic form $\Delta = \frac{1}{2}\sum_{n,m}C_{nm}f_n f_m$, which generates operators in the quadratic subalgebra
\begin{equation}
    \mathcal{A}^{(2)} = \mathrm{span}\{I, q^2, p^2, qp+pq, q, p\}.
    \label{eq:HO_quadratic_algebra}
\end{equation}
This is a 6-dimensional space (5 free parameters excluding normalisation). The HMF is therefore a general quadratic polynomial:
\begin{equation}
    \HMF = \frac{p^2}{2m_{\mathrm{MF}}} + \frac{m_{\mathrm{MF}}\omega_{\mathrm{MF}}^2}{2}q^2 + \gamma(qp+pq) + \epsilon_q q + \epsilon_p p + c_0 I.
    \label{eq:HMF_HO}
\end{equation}
This reproduces the known result: the Caldeira--Leggett reduced equilibrium state is Gaussian~\cite{grabertQuantumBrownianMotion1988,hiltHamiltonianMeanForce2011}, and the HMF is a renormalised harmonic Hamiltonian with possible squeezing ($\gamma$) and displacement ($\epsilon_q, \epsilon_p$) terms. The renormalised mass $m_{\mathrm{MF}}$ and frequency $\omega_{\mathrm{MF}}$ are determined by the Matsubara weights $\kappa_\ell$ and generating-function moments $I_n(\nu_\ell)$.

\paragraph{Non-Gaussian bath.}
For a non-Gaussian bath, the fourth cumulant generates $\Phi^{(4)} \propto q^4$ plus commutator-dressed terms ($q^3 p$, $q^2 p^2$, etc.), and the master algebra expands to include quartic and higher-order operators. Since $\dim\mathcal{H}_S = \infty$, the Cayley--Hamilton closure does not apply, and the polynomial hierarchy does not terminate. In this sense, the harmonic oscillator is the simplest system where the master-algebra closure is \emph{not} guaranteed by finite dimensionality. Closure must be imposed as a physical assumption (e.g.\ Gaussianity of the bath) rather than derived from the algebra.

This example illustrates a key limitation of the framework: for infinite-dimensional systems, the closure theorem (Theorem~\ref{thm:closure}) does not apply, and additional physical input is needed to determine whether $\HMF$ has a finite operator representation.

\subsection{Many-body spin chain with approximate symmetry}
\label{sec:example_spin_chain}

Consider an $N$-site Heisenberg XXZ chain
\begin{equation}
    H_S = J\sum_{i=1}^{N-1}\left(\sigma_x^{(i)}\sigma_x^{(i+1)} + \sigma_y^{(i)}\sigma_y^{(i+1)} + \Delta_z\sigma_z^{(i)}\sigma_z^{(i+1)}\right),
    \label{eq:XXZ_chain}
\end{equation}
coupled to a bath through $f = S_z = \frac{1}{2}\sum_i \sigma_z^{(i)}$.

\paragraph{Symmetry-protected closure.}
Since $[H_S, S_z] = 0$ for any $\Delta_z$, the adjoint chain is trivial: $f_n = 0$ for $n\geq 1$. The master algebra reduces to $\mathcal{A}_f = \mathrm{span}\{I, S_z, S_z^2, \ldots\}$. The Cayley--Hamilton theorem for $S_z$ (which has eigenvalues $\{-N/2, -N/2+1, \ldots, N/2\}$, at most $N+1$ distinct) gives closure at $\dim\mathcal{A}_f = N+1$.

The HMF is a polynomial in $S_z$ alone:
\begin{equation}
    \HMF = H_S - \frac{1}{\beta}\sum_{n=1}^{N}\alpha_{2n}\,S_z^{2n} + c_0 I.
    \label{eq:HMF_XXZ_Sz}
\end{equation}
This is exact for any bath (Gaussian or not). The bath can renormalise the Ising anisotropy (via $S_z^2$), introduce effective quartic interactions ($S_z^4$), etc., but it \emph{cannot} break the $U(1)$ symmetry or generate off-diagonal (spin-flip) operators. Integrability of the XXZ chain, if present, is preserved by $\HMF$.

\paragraph{Symmetry-breaking coupling.}
Now suppose the coupling has a small transverse component: $f = S_z + \epsilon\,\sigma_x^{(1)}$ for small $\epsilon$. The commutator $[H_S, \sigma_x^{(1)}]$ generates operators on sites 1 and 2 (through the nearest-neighbour coupling), and iterated commutators spread operators to all $N$ sites. The adjoint chain closure order grows with $N$, and the master algebra can become exponentially large ($D \sim 4^N$) for generic $\epsilon$.

However, the structure of the commutator expansion provides a natural perturbative hierarchy: at order $\epsilon^k$, only operators with support on at most $k+1$ contiguous sites contribute. This suggests a truncation strategy: retain the $k$-local subalgebra of $\mathcal{A}_{f,H_S}$ and truncate at a fixed locality. The non-uniqueness result (Section~\ref{sec:non_uniqueness}) guarantees that this truncation is not universally optimal, but for small $\epsilon$ it is the natural choice.

\paragraph{Beyond state of the art.}
The explicit construction of $\HMF$ as a polynomial in $S_z$ for the exactly commuting case, Eq.~\eqref{eq:HMF_XXZ_Sz}, goes beyond existing results in several ways:
\begin{enumerate}
    \item It is valid for \emph{arbitrary} bath statistics, not just Gaussian, and provides the exact operator content without any weak-coupling or Markovian assumption.
    \item The polynomial coefficients $\{\alpha_{2n}\}$ can be computed from the bath cumulants in closed form using the nested HS construction.
    \item The preservation of integrability is a structural consequence of the algebra, not an approximation: the HMF is a function of a conserved charge, so any Bethe-ansatz solution of $H_S$ remains valid for $\HMF$ with renormalised parameters.
\end{enumerate}

\section{Discussion}
\label{sec:discussion}

\subsection{What rules equilibrium: a summary}

The equilibrium of an open quantum system is governed by a single algebraic object: the master algebra $\mathcal{A}_{f,H_S}$, generated by the coupling operator $f$ and the derivation $\adf(\cdot) = [H_S,\cdot]$. This algebra has two independent sources of operator content --- non-Gaussianity and non-commutativity --- and the interplay between them determines the full structure of the Hamiltonian of mean force.

For finite-dimensional systems, the algebra always closes (Theorem~\ref{thm:closure}), guaranteeing a finite-parameter representation of $\HMF$. The dimension of the algebra provides an \emph{a priori} bound on the number of independent parameters the bath can introduce. This bound is tight: any operator in $\mathcal{A}_{f,H_S}$ can be realised as $\HMF$ for some choice of bath. The algebra is therefore both an upper bound on and a complete characterisation of the accessible operator content.

\subsection{The Caldeira--Leggett model as a representational choice}

A recurring theme is that the standard Caldeira--Leggett model --- a harmonic bath with bilinear coupling --- is not a physical necessity but a choice of representation. The Gaussian assumption eliminates the polynomial hierarchy (only the second cumulant contributes), while the bilinear coupling determines the system operator $f$ and hence the adjoint chain.

The quantum--classical equivalence theorem (Theorem~\ref{thm:qc_equivalence}) makes this concrete: any equilibrium state produced by the CL model can also be produced by a classical non-Gaussian bath with commuting coupling. The CL model is convenient because it simultaneously yields tractable dynamics (Gaussian baths admit exact path-integral methods~\cite{feynmanTheoryGeneralQuantum1963a,caldeiraQuantumTunnellingDissipative1983a}), but its equilibrium content is not unique.

This observation has practical consequences. Much of the strong-coupling literature debates whether specific corrections to $\HMF$ are ``real'' or artifacts of the CL assumption. The master-algebra framework resolves this: the corrections are real \emph{as elements of the algebra}, but their coefficients depend on the representation (which bath model is assumed). Two different bath models can produce the same equilibrium state with different decompositions into algebraic generators.

\subsection{Why methods disagree: a structural explanation}

The non-uniqueness result (Proposition~\ref{prop:truncation_inequivalence}) provides a satisfying resolution to a longstanding puzzle. Weak-coupling perturbation theory, polaron transformations, Born--Markov theory, reaction coordinate mappings, and variational methods all produce different approximations to $\HMF$ in the same parameter regime. This is not a failure of any one method but a mathematical inevitability: they truncate different exact series, and no truncation dominates all others.

The practical lesson is that the choice of approximation method should be guided by the structure of the master algebra for the problem at hand. If the algebra is small (e.g.\ a qubit with $D=4$), any method will work. If the algebra is large but has a natural block structure (e.g.\ a spin chain with conserved $S_z$), methods that respect the symmetry will outperform those that do not. If no symmetry is available, the dimension count $D$ provides a measure of the problem's intrinsic complexity.

\subsection{Implications for strong-coupling thermodynamics}

The framework has direct implications for the strong-coupling thermodynamic program~\cite{seifertFirstSecondLaw2016,jarzynskiNonequilibriumWorkTheorem2004,talknerColloquiumStatisticalMechanics2020}:

\begin{enumerate}
    \item \textbf{Thermodynamic potentials.} Since $\HMF$ is determined by finitely many parameters (for finite $d$), the free energy, entropy, and specific heat of the reduced system are functions of finitely many bath-dependent coefficients. Their strong-coupling corrections can be organised systematically in the algebra basis.
    
    \item \textbf{Heat and work.} The definitions of heat and work at strong coupling~\cite{espositoNatureHeatStrongly2015,rivasStrongCouplingThermodynamics2020} depend on choosing a decomposition of $\HMF$. The non-uniqueness result implies that this decomposition is representation-dependent, adding a layer of convention to the already debated question of energy partitioning.
    
    \item \textbf{Fluctuation relations.} The Jarzynski equality and Crooks relations hold for any $\HMF$ and do not depend on the representation~\cite{jarzynskiNonequilibriumWorkTheorem2004,campisiFluctuationTheoremArbitrary2009}. They are therefore robust against the non-uniqueness identified here.
\end{enumerate}

\subsection{Open questions}

Several natural extensions remain.

\paragraph{Dynamics.}
The quantum--classical equivalence is strictly an equilibrium result. The real-time dynamics of a system coupled to a quantum bath and a classical non-Gaussian bath generically differ, because the Schwinger--Keldysh influence functional depends on the full analytic structure of the bath correlations, not just their Euclidean moments. An analogous dynamical duality would require matching not just the static cumulants but the full time-dependent correlation functions --- a much stronger condition that is unlikely to hold generically. Elucidating the precise boundary between equilibrium equivalence and dynamical inequivalence is an important open problem.

\paragraph{Infinite-dimensional systems.}
For harmonic oscillators and field theories, the Cayley--Hamilton closure fails and the master algebra is generically infinite-dimensional. Gaussian baths still produce tractable quadratic $\HMF$, but non-Gaussian baths generate unbounded polynomial hierarchies. Characterising the conditions under which a finite truncation is controlled (e.g.\ renormalisability conditions) is a natural next step.

\paragraph{Multipartite coupling.}
For systems with multiple coupling operators ($H_I = \sum_\alpha f_\alpha\otimes B_\alpha$), the master algebra is generated by the union $\{(f_\alpha)_n\}$, and its dimension can be significantly larger than for single-operator coupling. The classification and the quantum--classical equivalence extend straightforwardly, but the matching conditions become more constrained and the examples richer.

\paragraph{Non-equilibrium steady states.}
The master-algebra framework is tailored to thermal equilibrium. For systems driven out of equilibrium (e.g.\ by multiple baths at different temperatures), the Hamiltonian of mean force is not defined, and the analogous question --- what determines the non-equilibrium steady state? --- requires a different formalism. Nevertheless, the algebraic structure of the operator content may still constrain the accessible steady states.

\subsection{Conclusion}

What rules equilibrium? The answer is an algebra. For any finite-dimensional system coupled to any bath through any single coupling operator, the Hamiltonian of mean force lies in the master algebra $\mathcal{A}_{f,H_S}$ --- a finite-dimensional, computable operator space determined by the system Hamiltonian and the coupling. Two sources feed this algebra: the polynomial hierarchy from non-Gaussian bath cumulants, and the commutator hierarchy from $[H_S,f]\neq 0$. Neither source alone tells the full story; neither is unique; and they are interchangeable at equilibrium. This algebraic perspective transforms the Hamiltonian of mean force from an opaque logged trace into a structured object whose content, dimension, and representability can be classified in advance --- before any bath model is specified and before any approximation is made.


\appendix
\section{Derivation of the Commutator Series}
\label{app:commutator_derivation}

This appendix provides the exhaustive derivation of the commutator-series representation for the Hamiltonian of mean force, as summarized in Sec.~\ref{sec:commutator_series}. We proceed in three steps: (1) exact elimination of the Gaussian bath to obtain a bilocal influence functional; (2) diagonalization of the kernel in Matsubara modes; and (3) inversion of the resulting product average via the Baker--Campbell--Hausdorff (BCH) formula.

\subsection{Gaussian Elimination and the Bilocal Form}
We consider the reduced equilibrium operator $\rhobar(\beta) = \Tr_B e^{-\beta H_{\mathrm{tot}}}$. Working in the interaction picture with respect to $H_0 = H_S + H_B$, the propagator can be written as
\begin{equation}
    \rhobar(\beta) = e^{-\beta H_S} \left\langle \mathcal{T}_\tau \exp\left( -\int_0^\beta d\tau \tilde{H}_I(\tau) \right) \right\rangle_B,
\end{equation}
where $\tilde{H}_I(\tau) = e^{\tau H_0} H_I e^{-\tau H_0} = \tilde{f}(\tau) \otimes \tilde{B}(\tau)$. For a Gaussian bath, the average over $B$ can be performed exactly using the cumulant expansion (Wick's theorem). Since odd cumulants vanish, the result is determined entirely by the second cumulant (the autocorrelation function):
\begin{equation}
    K(\tau-\sigma) = \langle \mathcal{T}_\tau \tilde{B}(\tau)\tilde{B}(\sigma) \rangle_B.
\end{equation}
The influence functional exponentiates to a quadratic form:
\begin{equation}
    \begin{split}
    \rhobar(\beta) &= e^{-\beta H_S} \mathcal{T}_\tau \exp\biggl( \frac{1}{2}\int_0^\beta d\tau \int_0^\beta d\sigma \\
    &\quad \times K(\tau-\sigma) \tilde{f}(\tau)\tilde{f}(\sigma) \biggr).
    \end{split}
    \label{eq:app_rhobar_Texp}
\end{equation}
We define the \emph{bilocal influence operator} $\Delta$ as the exponent:
\begin{equation}
    \Delta \equiv \frac{1}{2}\int_0^\beta d\tau \int_0^\beta d\sigma K(\tau-\sigma) \tilde{f}(\tau)\tilde{f}(\sigma).
    \label{eq:app_Delta_def}
\end{equation}
Crucially, although Eq.~\eqref{eq:app_rhobar_Texp} formally retains time-ordering, the operator $\Delta$ itself contains all the time-dependence integrated out. As we show below, $\Delta$ can be expressed purely in terms of time-independent operators, making the time-ordering redundant. Thus $\rhobar(\beta) = e^{-\beta H_S}e^\Delta$.

\subsection{Adjoint Chain and Matsubara Expansion}
The interaction-picture coupling $\tilde{f}(\tau) = e^{\tau H_S} f e^{-\tau H_S}$ is generated by the adjoint action of $H_S$. Defining the adjoint chain $f_n = \mathrm{ad}_{H_S}^n(f)$, we have the expansion
\begin{equation}
    \tilde{f}(\tau) = \sum_{n=0}^\infty \frac{\tau^n}{n!} f_n.
\end{equation}
Substituting this into Eq.~\eqref{eq:app_Delta_def}, we can perform the time integrals explicitly. Using the Matsubara expansion of the translation-invariant kernel $K(u) = \beta^{-1} \sum_\ell \kappa_\ell e^{i\nu_\ell u}$, the double integral factorizes:
\begin{align}
    \Delta &= \frac{1}{2\beta} \sum_{\ell \in \mathbb{Z}} \kappa_\ell \left( \int_0^\beta d\tau e^{i\nu_\ell \tau} \tilde{f}(\tau) \right) \left( \int_0^\beta d\sigma e^{-i\nu_\ell \sigma} \tilde{f}(\sigma) \right) \nonumber \\
    &= \frac{1}{2\beta} \sum_{\ell \in \mathbb{Z}} \kappa_\ell \tilde{F}_\ell \tilde{F}_\ell^\dagger,
\end{align}
where $\tilde{F}_\ell = \sum_{n=0}^\infty I_n(\nu_\ell) f_n$ are the mode-projected operators, and $I_n(\nu_\ell)$ are the moments of the time integration.
Collecting terms by chain index $n,m$, we obtain the representation
\begin{equation}
    \Delta = \frac{1}{2} \sum_{n,m=0}^\infty C_{nm}(\beta) f_n f_m,
    \label{eq:app_Delta_Cnm}
\end{equation}
where $C_{nm}(\beta)$ encodes the bath spectral information. Since $f_n$ are time-independent operators, $\Delta$ is indeed a standard operator on $\mathcal{H}_S$.

\subsection{BCH Inversion and the Bernoulli Series}
The HMF is defined by $e^{-\beta \HMF} = \rhobar(\beta) = e^{-\beta H_S}e^\Delta$. We extract $\HMF$ using the Baker--Campbell--Hausdorff (BCH) formula for $\log(e^A e^B)$ with $A=-\beta H_S$ and $B=\Delta$:
\begin{equation}
    \begin{split}
    -\beta \HMF &= -\beta H_S + \Delta + \frac{1}{2}[-\beta H_S, \Delta] \\
    &\quad + \frac{1}{12}[-\beta H_S, [-\beta H_S, \Delta]] + \dots
    \end{split}
\end{equation}
To linear order in the perturbation $\Delta$, this series can be resummed exactly using the generating function of the Bernoulli numbers. The operation $X \mapsto [A, X]$ is denoted by $\mathrm{ad}_A$. The linear-in-$B$ part of the BCH series is $\frac{\mathrm{ad}_A}{1-e^{-\mathrm{ad}_A}} B$.
Thus, identifying $\mathrm{ad}_A = -\beta \mathrm{ad}_{H_S}$, we have
\begin{equation}
    \HMF = H_S - \frac{1}{\beta} \frac{-\beta \mathrm{ad}_{H_S}}{1-e^{\beta \mathrm{ad}_{H_S}}} (\Delta) + \mathcal{O}(\Delta^2).
\end{equation}
Using the identity $\frac{x}{e^x-1} = \sum_{r=0}^\infty \frac{B_r}{r!} x^r$ and $x/(1-e^{-x}) = x + x/(e^x-1)$, we obtain the explicit expansion:
\begin{equation}
    \HMF = H_S - \frac{1}{\beta} \sum_{r=0}^\infty \frac{B_r}{r!} (-\beta)^r \mathrm{ad}_{H_S}^r(\Delta) + \mathcal{O}(\Delta^2).
\end{equation}
Substituting the explicit form of $\Delta$ from Eq.~\eqref{eq:app_Delta_Cnm} and using the binomial identity for the adjoint action on a product, $\mathrm{ad}^r(f_n f_m) = \sum_k \binom{r}{k} f_{n+k} f_{m+r-k}$, yields the fully explicit commutator series:
\begin{align}
    \HMF &= H_S - \frac{1}{2\beta} \sum_{r=0}^\infty \frac{B_r}{r!} (-\beta)^r \sum_{n,m=0}^\infty C_{nm}(\beta) \nonumber \\
    &\quad \times \sum_{k=0}^r \binom{r}{k} f_{n+k} f_{m+r-k} + \mathcal{O}(\Delta^2).
\end{align}
This result shows that $\HMF$ is composed of linear combinations of products of algebra elements $f_n$, with coefficients determined by the bath statistics ($C_{nm}$) and the combinatorial structure of the BCH series ($B_r$).


\bibliography{../../literature/references}

\end{document}
