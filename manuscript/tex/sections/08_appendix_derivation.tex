\section{Derivation of the Commutator Series}
\label{app:commutator_derivation}

This appendix provides the exhaustive derivation of the commutator-series representation for the Hamiltonian of mean force, as summarized in Sec.~\ref{sec:commutator_series}. We proceed in three steps: (1) exact elimination of the Gaussian bath to obtain a bilocal influence functional; (2) diagonalization of the kernel in Matsubara modes; and (3) inversion of the resulting product average via the Baker--Campbell--Hausdorff (BCH) formula.

\subsection{Gaussian Elimination and the Bilocal Form}
We consider the reduced equilibrium operator $\rhobar(\beta) = \Tr_B e^{-\beta H_{\mathrm{tot}}}$. Working in the interaction picture with respect to $H_0 = H_S + H_B$, the propagator can be written as
\begin{equation}
    \rhobar(\beta) = e^{-\beta H_S} \left\langle \mathcal{T}_\tau \exp\left( -\int_0^\beta d\tau \tilde{H}_I(\tau) \right) \right\rangle_B,
\end{equation}
where $\tilde{H}_I(\tau) = e^{\tau H_0} H_I e^{-\tau H_0} = \tilde{f}(\tau) \otimes \tilde{B}(\tau)$. For a Gaussian bath, the average over $B$ can be performed exactly using the cumulant expansion (Wick's theorem). Since odd cumulants vanish, the result is determined entirely by the second cumulant (the autocorrelation function):
\begin{equation}
    K(\tau-\sigma) = \langle \mathcal{T}_\tau \tilde{B}(\tau)\tilde{B}(\sigma) \rangle_B.
\end{equation}
The influence functional exponentiates to a quadratic form:
\begin{equation}
    \begin{split}
    \rhobar(\beta) &= e^{-\beta H_S} \mathcal{T}_\tau \exp\biggl( \frac{1}{2}\int_0^\beta d\tau \int_0^\beta d\sigma \\
    &\quad \times K(\tau-\sigma) \tilde{f}(\tau)\tilde{f}(\sigma) \biggr).
    \end{split}
    \label{eq:app_rhobar_Texp}
\end{equation}
We define the \emph{bilocal influence operator} $\Delta$ as the exponent:
\begin{equation}
    \Delta \equiv \frac{1}{2}\int_0^\beta d\tau \int_0^\beta d\sigma K(\tau-\sigma) \tilde{f}(\tau)\tilde{f}(\sigma).
    \label{eq:app_Delta_def}
\end{equation}
Crucially, although Eq.~\eqref{eq:app_rhobar_Texp} formally retains time-ordering, the operator $\Delta$ itself contains all the time-dependence integrated out. As we show below, $\Delta$ can be expressed purely in terms of time-independent operators, making the time-ordering redundant. Thus $\rhobar(\beta) = e^{-\beta H_S}e^\Delta$.

\subsection{Adjoint Chain and Matsubara Expansion}
The interaction-picture coupling $\tilde{f}(\tau) = e^{\tau H_S} f e^{-\tau H_S}$ is generated by the adjoint action of $H_S$. Defining the adjoint chain $f_n = \mathrm{ad}_{H_S}^n(f)$, we have the expansion
\begin{equation}
    \tilde{f}(\tau) = \sum_{n=0}^\infty \frac{\tau^n}{n!} f_n.
\end{equation}
Substituting this into Eq.~\eqref{eq:app_Delta_def}, we can perform the time integrals explicitly. Using the Matsubara expansion of the translation-invariant kernel $K(u) = \beta^{-1} \sum_\ell \kappa_\ell e^{i\nu_\ell u}$, the double integral factorizes:
\begin{align}
    \Delta &= \frac{1}{2\beta} \sum_{\ell \in \mathbb{Z}} \kappa_\ell \left( \int_0^\beta d\tau e^{i\nu_\ell \tau} \tilde{f}(\tau) \right) \left( \int_0^\beta d\sigma e^{-i\nu_\ell \sigma} \tilde{f}(\sigma) \right) \nonumber \\
    &= \frac{1}{2\beta} \sum_{\ell \in \mathbb{Z}} \kappa_\ell \tilde{F}_\ell \tilde{F}_\ell^\dagger,
\end{align}
where $\tilde{F}_\ell = \sum_{n=0}^\infty I_n(\nu_\ell) f_n$ are the mode-projected operators, and $I_n(\nu_\ell)$ are the moments of the time integration.
Collecting terms by chain index $n,m$, we obtain the representation
\begin{equation}
    \Delta = \frac{1}{2} \sum_{n,m=0}^\infty C_{nm}(\beta) f_n f_m,
    \label{eq:app_Delta_Cnm}
\end{equation}
where $C_{nm}(\beta)$ encodes the bath spectral information. Since $f_n$ are time-independent operators, $\Delta$ is indeed a standard operator on $\mathcal{H}_S$.

\subsection{BCH Inversion and the Bernoulli Series}
The HMF is defined by $e^{-\beta \HMF} = \rhobar(\beta) = e^{-\beta H_S}e^\Delta$. We extract $\HMF$ using the Baker--Campbell--Hausdorff (BCH) formula for $\log(e^A e^B)$ with $A=-\beta H_S$ and $B=\Delta$:
\begin{equation}
    \begin{split}
    -\beta \HMF &= -\beta H_S + \Delta + \frac{1}{2}[-\beta H_S, \Delta] \\
    &\quad + \frac{1}{12}[-\beta H_S, [-\beta H_S, \Delta]] + \dots
    \end{split}
\end{equation}
To linear order in the perturbation $\Delta$, this series can be resummed exactly using the generating function of the Bernoulli numbers. The operation $X \mapsto [A, X]$ is denoted by $\mathrm{ad}_A$. The linear-in-$B$ part of the BCH series is $\frac{\mathrm{ad}_A}{1-e^{-\mathrm{ad}_A}} B$.
Thus, identifying $\mathrm{ad}_A = -\beta \mathrm{ad}_{H_S}$, we have
\begin{equation}
    \HMF = H_S - \frac{1}{\beta} \frac{-\beta \mathrm{ad}_{H_S}}{1-e^{\beta \mathrm{ad}_{H_S}}} (\Delta) + \mathcal{O}(\Delta^2).
\end{equation}
Using the identity $\frac{x}{e^x-1} = \sum_{r=0}^\infty \frac{B_r}{r!} x^r$ and $x/(1-e^{-x}) = x + x/(e^x-1)$, we obtain the explicit expansion:
\begin{equation}
    \HMF = H_S - \frac{1}{\beta} \sum_{r=0}^\infty \frac{B_r}{r!} (-\beta)^r \mathrm{ad}_{H_S}^r(\Delta) + \mathcal{O}(\Delta^2).
\end{equation}
Substituting the explicit form of $\Delta$ from Eq.~\eqref{eq:app_Delta_Cnm} and using the binomial identity for the adjoint action on a product, $\mathrm{ad}^r(f_n f_m) = \sum_k \binom{r}{k} f_{n+k} f_{m+r-k}$, yields the fully explicit commutator series:
\begin{align}
    \HMF &= H_S - \frac{1}{2\beta} \sum_{r=0}^\infty \frac{B_r}{r!} (-\beta)^r \sum_{n,m=0}^\infty C_{nm}(\beta) \nonumber \\
    &\quad \times \sum_{k=0}^r \binom{r}{k} f_{n+k} f_{m+r-k} + \mathcal{O}(\Delta^2).
\end{align}
This result shows that $\HMF$ is composed of linear combinations of products of algebra elements $f_n$, with coefficients determined by the bath statistics ($C_{nm}$) and the combinatorial structure of the BCH series ($B_r$).
